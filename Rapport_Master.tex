\documentclass[a4paper,11pt]{report}
\usepackage[T1]{fontenc} % pour écrire en français
\usepackage[francais]{babel} %pour écrire en français
\usepackage[utf8x]{inputenc} %encodage en UTF-8
\usepackage{fancyhdr} %pour gérer les en-têtes et pieds de page
\usepackage{amsmath,amscd,amssymb} %pour insérer des expressions scientifiques
\usepackage[pdftex]{graphicx} %pour inclure des figures
\usepackage{subfig}
\usepackage{hyperref} %pour créer des liens hyper-textes
\usepackage{verbatim} %pour citer du code Latex ou autre
\usepackage{url} %pour citer une adresse web
\pagenumbering{arabic} %type de numérotation des pages
\graphicspath{{Figures/}} %les figures sont rangées dans le dossier Figures
\pagestyle{plain} %style des pages

%%%%%%%%%%%%%%%%%%%%%%%%%%%%%%%%%%%%%%%%%%%%%%%%%%%%%%%%%%%%%%%%%%%%

\begin{document}

%----------------------------------------------------------------------------------------------------------
%				PAGE DE GARDE
%----------------------------------------------------------------------------------------------------------
\makeatletter
\def\clap#1{\hbox to 0pt{\hss #1\hss}}%
\def\ligne#1{%
\hbox to \hsize{%
\vbox{\centering #1}}}%
\def\haut#1#2#3{%
\hbox to \hsize{%
\rlap{\vtop{\raggedright #1}}%
\hss
\clap{\vtop{\centering #2}}%
\hss
\llap{\vtop{\raggedleft #3}}}}%
\def\bas#1#2#3{%
\hbox to \hsize{%
\rlap{\vbox{\raggedright #1}}%
\hss
\clap{\vbox{\centering #2}}%
\hss
\llap{\vbox{\raggedleft #3}}}}%
\def\maketitle{%
\thispagestyle{empty}\vbox to \vsize{%
\haut{}{\@blurb}{}
\vfill
\vspace{1cm}
\begin{flushleft}
\usefont{OT1}{ptm}{m}{n}
\huge \@title
\end{flushleft}
\par
\hrule height 4pt
\par
\begin{flushright}
\usefont{OT1}{phv}{m}{n}
\Large \@author
\par
\end{flushright}
\vspace{3.5cm}
\centering
\includegraphics[width=0.4\textwidth]{Photo_univlemans} \hfill
\vfill
\bas{}{\@location}{}
}%
\cleardoublepage
}
\def\author#1{\def\@author{#1}}
\def\title#1{\def\@title{#1}}
\def\location#1{\def\@location{#1}}
\def\blurb#1{\def\@blurb{#1}}
\author{}
\title{}
\location{Le Mans}\blurb{}
\makeatother
\title{Synchronisation Mutuelle des M\'etronomes}
\author{Tony Merrien \& R\'emi Plantade}
\location{Le Mans, Année Universitaire 2014-2015}
\blurb{%
Université du Maine\\
Faculté des Sciences
\textbf{Rapport de projet}\\[1em]
Semestre 3 L2 SPI\\
Encadrant : Guillaume Penelet
}%

%%%%%%%%%%%%%%%%%%%%%%%%%%%%%%%%%%%%%%%%%%%%%%%%%%%%%%%%%%%%%%%%%%%%%%%%%%%%%%%%%%%%%%%%%%%%%%%%%%%%%%%%%%%%%%%%


\maketitle  %génère la page de garde

\newpage  %comme son nom l'indique ...
\pagenumbering{roman} \setcounter{page}{1} %les pages commencent à être numérotées en lettre romaines.

%%%%%%%%%%%%%%%%%%%%%%%%%%%%%%%%%%%%%%%%%%%%%%%%%%%%%%%%%

\newpage

 %------------------------------------------------------------------------------------------------------
 %					    TABLE DES MATIÈRES 
 %-----------------------------------------------------------------------------------------------------

{\tableofcontents} 
%\newpage\
\listoffigures


\newpage


%%%%%%%%%%%%%%%%%%%%%%%%%%%%%%%%%%%%%%%%%%%%%%%%%%%%%%%%%%%%%%%%%%%%%%%%%%%%%%%%%%%%%%%%%%%%%%%%%%%%%%%%%%%%%%%%
\chapter*{Introduction}
\addcontentsline{toc}{chapter}{Introduction} %Ce chapitre n'est pas numéroté mais apparaitra dans la table des matières gràce à cette commande.
\pagenumbering{arabic} \setcounter{page}{1} %Le numéro de page est remis à zéro et la numérotation est en chiffre arabe

 La synchronisation est un phénomène universellement observable et étudié depuis des siècles. Une foule qui applaudit, des personnes marchant côte à côte, les pendules de Huygens et les métronomes que nous allons étudier dans l'ensemble de ce projet sont les conséquences directement observable de ce phénomène physique. Découvert dans les années 1665 par Christiann Huygens alors qu'il cherchai à aider les hommes partant en mer pour de longue période afin de ne jamais perdre le fil du temps. Il constata que deux pendules solidaires d'un support finnissaient toujours par se synchroniser. Nous allons ici étudier plus en détail ce phénomène, à l'aide de constatation visuelle, de mesures sonores et de travaux théoriques pour différentes conditions expérimentales.

%%%%%%%%%%%%%%%%%%%%%%%%%%%%%%%%%%%%%%%%%%%%%%%%%%%%%%%%%%%%%%%%%%%%%%%%%%%%%%%%%%%%%%%%%%

\chapter{La synchronisation}
%%%%%%%%%%%%%%%%%%%%%%%%%%%%%%%%%%%%%%%%%%%%%%%%%%%%%%%%%%%%%%%%%%%%%%%%%%%%%%%%%%%%%%%%%%
\section{Définitions}

Synchronisation (universelle) :

Synchronisation (scientifique) : 

\subsection{}


\subsection{}


\subsection{}


%%%%%%%%%%%%%%%%%%%%%%%%%%%%%%%%%%%%%%%%%%%%%%%%%%%%%%%%%%%%%%%%%%%%%%%%%%%%%%%%%%%%%%%%%%

%bibliography
\renewcommand{\bibname}{Références} 
\begin{thebibliography}{2}
\section*{Bibliographie}
  % \bibitem[label]{cle} Auteur, TITRE, editeur, annee
   \bibitem{lam1} PIKOVSKY, {\it Synchronisation, A Non Linear Science, REF, 1994}
  \bibitem{chev} C. CHEVALIER, {\it \LaTeX{} pour l'impatient , H\&K, 2009} 
  \bibitem{maq} N.-A. MAQUIS, {\it Rédigez des documents de qualité avec \LaTeX , Livre du zéro, 2010} 
\section*{Webographie}
\bibitem{grappa} \url{http://www.grappa.univ-lille3.fr/FAQ-LaTeX/}
\bibitem{wiki} \url{http://fr.wikipedia.org/wiki/Synchronisation}
\bibitem{symbol} \url{http://amath.colorado.edu/documentation/LaTeX/Symbols.pdf}
\end{thebibliography}

%%%%%%%%%%%%%%%%%%%%%%%%%%%%%%%%%%%%%%%%%%%%%%%%%%%%%%%%%%%%%%%%%%%%%%%%%%%%%%%%%%%%%%%%%%
\chapter*{Conclusion}
%%%%%%%%%%%%%%%%%%%%%%%%%%%%%%%%%%%%%%%%%%%%%%%%%%%%%%%%%%%%%%%%%%%%%%%%%%%%%%%%%%%%%%%%%%
\addcontentsline{toc}{chapter}{Conclusion} %permet d'inclure le chapitre conclusion dans la table des matières

La synchronisation...

%%%%%%%%%%%%%%%%%%%%%%%%%%%%%%%%%%%%%%%%%%%%%

\appendix 
\chapter{Programmes Matlab}
\label{garde}

\begin{verbatim}



\end{verbatim}}
 
%%%%%%%%%%%%%%%%%%%%%%%%%%%%%%%%%%%%%%%%%%%%%
\end{document}