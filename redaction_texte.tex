

limites de synchronisation

	Après avoir étudié et caractérisé le matériel mise à notre disposition, l'étude du phénomène est entamé avec différentes expériences. La synchronisation de métronomes ayant été déjà traité, les premières expériences ont pour but de vérifier les informations obtenus dans ce rapport afin, par la suite, de pousser l'étude au delà.
	 
	Lors de la première expérience avec notre banc de mesure, la limite de fréquence des deux métronomes pour qu'il y est synchronisation est recherchée. Pour cela les deux métronomes sont réglés, par le biais de la position de la masselotte sur la tige pour obtenir un battement par minute (BPM) faible, ce qui signifie que la fréquence est basse, environ 50 BPM pour les deux lors de cette première manipulation. Le phénomène ne se produisant pas les métronomes sont réglés sur une autre fréquence plus élevé et la manipulation est renouvelée. Ainsi avec cette expérience la détermination d'une limite de fréquence pour l'observation du phénomène de synchronisation est obtenue. Les résultats acquis sont en accord avec ceux du rapport des étudiants [bibliographie, page 7 8]. En effet la synchronisation entre deux métronomes ne s'effectue pas en basse fréquence et un seuil a été déterminé, qui se situe à 176 BPM. La limite de déplacement de la massette sur le métronome empêche la détermination d'un éventuel seuil de haute fréquence où le phénomène ne se produit plus. Au cours de l'expérience, lorsque les deux métronomes se synchronisent ils sont en opposition de phase. En effet lorsque les métronomes sont désynchronisés la plaque  de déplace sur les cannettes tandis que lorsque les deux métronomes sont synchroniser en opposition de phase la plaque reste immobile, le système est donc à son état d'équilibre. Le transfert d'énergie entre les deux métronomes et le système (plaque en bois et les deux cannettes) est émise ( a reformuler ).
	 




synchronisation pour différentes fréquences entre les deux métronomes

	A présent l'étude dans cette partie se porte sur la synchronisation pour une différence de fréquence entre les deux métronomes. Sachant que le phénomène commence à se produire pour au moins 176 BPM, Mr.Brown est réglé sur 180 BPM tandis que Mr.White sur un battement par minute plus faible afin de trouver la limite inférieur où il y a synchronisation. Les deux métronomes sont disposés sur le système afin de déterminer si il y a ou non synchronisation. Si le phénomène se produit alors les deux métronomes sont posés sur la table et leur fréquence est mesurés et traité par un programme. On réitère cette manipulation jusqu'à tomber sur la limite de la différence de fréquence entre les deux métronomes pour qu'il y est synchronisation.   Ensuite Mr.Brown reste fixé a 176 BPM et Mr.White est réglé pour une fréquence supérieure pour que de même que précédemment déterminer la limite mais cette fois si la limite supérieur de la synchronisation pour une différence de fréquences entre les deux métronomes. Cette procédure est répétée pour trouver, suivant la fréquence fixe, les limites et ainsi tracer une courbe afin d'en tiré une analyse du comportement.( a revoir la formulation). 
	
					TITRE !!!
					
					COURBE !!!
					
					LEGENDE !!!
					
					EXPLICATION DE LA COURBE OBTENUE !!!
					
Une étude est ensuite menée afin de savoir si il y a une synchronisation possible pour des fréquences harmoniques, c'est à dire un métronome fixé à une certaine fréquence f0 et l'autre métronome à un fréquence 2 fois plus ou moins élevé. La manipulation est réalisée avec Mr.Brown réglé à 208BPM et Mr.White à 104BPM. L'expérience n'est pas concluante. En effet précédemment un seuil de fréquence pour obtenir une synchronisation a été trouvé, 176 BPM , or ici Mr.White est réglé à 104BPM qui est bien inférieur au seuil. Les métronomes pouvant allés que jusqu'à 230BPM l'étude sur la synchronisation en régime harmonique est compromise. 




pot vibrant

Dans cette partie l'étude se porte sur la synchronisation des deux métronomes avec un pot vibrant relié à la plaque en bois. Le pot vibrant est un objet permettant d'exciter le système par le biais, ici, d'un pas de vis relié avec une vis à la plaque en bois. La fréquence d'excitation et son amplitude sont réglés avec l'amplificateur relié au pot vibrant.

												FORMULE POUR PASSER DES BPM AU HERTZ

 L'expérience réalisé ici consiste à donner à la plaque en bois une certaine fréquence d'excitation correspond à la fréquence d'oscillation du métronome suivant où est placé la massette et une amplitude d'oscillation au système(à reformuler).
Ainsi avec ce nouveau banc de mesure il est possible d'étudier si les limites de la synchronisation change et d'observer l'influence de l'amplitude sur la synchronisation.
 
